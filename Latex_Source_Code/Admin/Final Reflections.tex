\chapter*{Final Reflections}

My initial aim for this project was to explore quantum computing from its mathematical foundations, through to its logic, and into its applications, with a focus on quantum cryptography. This was a wide scope that I planned. It included concepts from superposition and entanglement to Shor's algorithm and even philosophical implications. This is evidenced by the evolution of my title from “Introduction to Quantum Computing” in the project plan, to “Evolution of Quantum Computing: Foundations, Practical Impacts and Challenges” in my draft, and then “The Mathematics of Quantum Theory \& Computing” in my final report. You can see as I got deeper into my research I realised, to go into sufficient depth on each point I would have to sacrifice elements. While editing I decided to prioritise a thorough mathematical foundation and the fundamental logic of quantum computing. I would then mention topics that I had cut in the conclusion, adding suggestions for further reading. However, I was unable to include all topics, one example is Shor's algorithm, while having impactful applications, it wasn't fit for the report due to it's complexity because of its reliance on quantum Fourier transforms. I cut philosophical implications as I was unable to find notable sources. 

\noindent A challenge I faced early in the process was the amount of mathematics in quantum theory I was unfamiliar with. Concepts like Tensor products, Hilbert spaces, Quantum Operators and partial traces were initially overwhelming. Early on, while I still wasn't comfortable with these concepts, I visited sources multiple times to fully understand. This led to a change in my approach, I started rephrasing definitions and arguments and trying out examples where I was less confident. This slowed my pace but helped me gain a deeper understanding of the content, through correcting mistakes or positive reinforcement.

\noindent A big moment came when I started researching entanglement. I already had an informal idea of entanglement, and in conjunction with my understanding of the mathematical representation of superposition, I thought if two quantum states formed a superposition and were expressible by a third, they were entangled. I realised this mistake as I did further research into entangled states. I found how critical tensor products are and I'd been unaware of the key idea of entanglement, separable vs non-separable states. This mistake led me down the path of loss of information and finding reduced density matrices. It also brought more new mathematics like partial traces. This challenge helped me gain a much deeper understanding of entanglement, a principle that would be fundamental going forward.

\noindent The research process deviated considerably from the project plan. Initially, I had planned on covering error correction and quantum cryptography, but to achieve this with sufficient detail I would have needed to go into group representation theory and quantum information theory. Instead, I expanded my knowledge of linear algebra and operator theory. Another significant pivot was when I decided to hone in on Grovers algorithm as I could explain it more succinctly and also use it to create an example Qiskit program simulating its operation.

\noindent I used feedback from my supervisor to improve the structure and accuracy of my report. For example, he informed me of the controversy surrounding Google's claims of quantum supremacy and IBM's criticism. The advice to keep my research relevant to what I will be covering later on, helped me minimise time wasted researching areas that didn't have relevance. The focus on Grover's algorithm over Shor's algorithm also came through discussion of realistic aims for the project.

\noindent This project has helped develop my mathematical research skills. I learnt that making mistakes is an important step for gaining a deeper understanding. My latex and time management skills improved by deconstructing tasks into small achievable goals and setting soft deadlines on myself.

\noindent If I did the project again, I would want to define the fundamentals of quantum computing and the mathematical foundations simultaneously, so that I'm able to go into detail on quantum teleportation with a Qiskit implementation and subsequently quantum encryption methods (quantum key encryption). After completing this project I feel I'm now better equipped for future research into areas I'm interested in like quantum machine learning and simulation.
