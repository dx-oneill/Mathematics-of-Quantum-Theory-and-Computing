\chapter{Conclusion}

This report formulated the key concepts of quantum theory bridging the gap from the abstract mathematical representation to in practice quantum computation. We started by introducing the linear algebra framework used to describe quantum states, superposition, entanglement and more. We then contextualised these fundamentals in the realm of quantum computing, describing qubits and how they are manipulated by quantum operators. Finally, we looked at the practical implementation via an example algorithm (Grover's search algorithm) and the use of Qiskit programming. 

\noindent One key idea we can take away is that quantum computing is superior under specific circumstances compared to its classical counterpart. The use of superposition, interference and entanglement enables significant speedups, but currently only for the certain algorithms/problems that take advantage of them.

\noindent Another point worth noting is, while there are accessible tools for working with quantum computers such as IBM's Qiskit and cloud quantum processors, there still doesn't exist a general purpose quantum computer that can be used for a range of applications. This is because of the numerous challenges that still exist in quantum computing. Currently, the cost to build and maintain quantum computers is very high, this is due to the massively difficult engineering needed for such a project. This stems from the environment quantum computers require to function as intended, which includes extreme conditions, such as near absolute zero temperatures and minimal disturbances, because of the fragility of qubits. This comes from trying to account for the issue of noise and decoherence (when the quantum system interacts with the environment losing its ability to utilise quantum phenomena). Decoherence is also a bottleneck for scalability as adding more qubits to the system magnifies the noise and decoherence causing the addition of each extra qubit to be exponentially complex. One technique that aims to counter decoherence is error correction, in source [1] it's emphasised that this is critical for increasing scalability. Today's machines still operate with a considerable level of noise, but during the time of writing this report there has been progress on this with the new topological architecture qubit made by Microsoft[31], that is stated to have massively larger scalability and fault tolerance potential.

\noindent There are many types of physical architectures of qubits as all that is essentially necessary is any two-level quantum system, common architectures include: ion traps, where the two-level system is the two energy levels of the ion, another is linear optical photons, where the two-level system is the two distinct optical
paths the photon can travel down. There are many more that we can't look into during this report, for more extensive detail you can refer to source [32].

\noindent Quantum computers are projected to make an impact in many fields ranging from materials science to cryptography. One example of a near future application is quantum key distribution (QKD) this a line of communication, provably safe from eavesdropping. There are already experiments being done using satellites to demonstrate QKD and quantum teleportation [33]. Another field with massive potential is quantum simulation, the capability that quantum computers possess in simulating quantum systems including modelling complex molecules is something classical computers can't match, [34] for further reading. There is also the question of what does AI look like when synergised with quantum computers? This is an area with massive potential for breakthroughs in drug discoveries and other areas, you can see this is interlinked with quantum simulation quite heavily source [35] goes into more detail on the topic.

\noindent We can conclude by saying quantum computing is a field that unifies mathematics, physics and computer science. It's technology that's directly developed from the linear algebra mathematics of quantum theory. While still not a fully realised field, there are rapid advancements happening and massive potential applications. It is a testament to how mathematics can drive technological revolutions.


