\chapter{Introduction}


\begin{quote}
\textit{"Nature isn't classical, dammit, and if you want to make a simulation of nature, you'd better make it quantum mechanical"}  \\
\hfill -- Richard Feynman
\end{quote}


\noindent Quantum computing represents the pinnacle of modern computation power and is the future of the field. As the scalability of quantum computing advances, we will see a shift in all technological infrastructures, with everyone trying to keep up with the tremendous leap in computing capabilities. Quantum computing harnesses the phenomena/principles of quantum mechanics, including; Superposition, Interference, and Entanglement, to create a new non-classical computer. Quantum computers hold the potential to solve problems previously thought to be unsolvable by classical computers. Proof of this potential was demonstrated when Google claimed quantum supremacy in 2019. In [1], quantum supremacy was described as "the era of quantum supremacy, when we will be able to perform tasks with controlled quantum systems going beyond what can be achieved with ordinary digital computers." in other words, quantum supremacy is the point in time when the computational abilities of a quantum computer surpass that of a classical computer. Google's claim of quantum supremacy was by their quantum processor 'Sycamore'. As detailed in [2], the feat was a computational problem, it was estimated it would have taken one of the world's fastest classical supercomputers 'Summit' 10,000 years to complete, and Sycamore finished it in 200 seconds. The proficiency of these machines is likely to revolutionize industries such as cryptography and cyber security, materials science, AI, optimization, and the pharmaceutical research industry. One point to note about this achievement of Google is that the problem was a program completing a hyper specific task that takes advantage of the alternative underpinning logic quantum computers use. Therefore it is often sensationalised, so we must remember to take into account that while this is a first step it is a long way off a general out performance of classical computers. To paraphrase [4], the biggest opposition to Google's claim of quantum supremacy came from IBM who disputed that the task would take at best 10,000 years on a classical machine. They claimed the same task can be simulated on a classical computer with an upper bound of 2.5 days. IBM also stated that the threshold of the original meaning of the term proposed by John Preskill had not been met.

\noindent To investigate the extent of the prospect of quantum computing this dissertation examines the concepts of quantum theory that make up quantum computing then going into quantum computing. The goal will be to develop an understanding of some key topics such as; quantum theory phenomena, quantum computing logic and circuits, and quantum algorithms and their impact. Another goal is to demonstrate a quantum program using quantum programming tools like Qiskit. These aims will also help us fully realise the potential of quantum computing.

\noindent We begin with the mathematics of quantum theory, this section will give a detailed description on the principles of quantum mechanics that underpin quantum computers and their physical implementation. This will develop a solid background for understanding the later topics. Following this we delve into quantum logic and algorithms, explaining how qubits differ from bits, leading into the difference between classical and quantum mechanical computing logic (quantum gates). We will study some quantum computing algorithms, for example, explaining how Grover's search algorithm operates. To finish that section tools for programming on quantum computers and simulation will be discussed, with example usage of Qiskit. This will be followed by a conclusion where we will reflect on the impact quantum computers could have on areas, including; quantum simulations, the philosophical implications of AI on quantum computers, and quantum cryptography. Challenges of quantum computing and the structure of qubits and quantum computers will also be touched on and sources to delve into these topics in detail will be provided. 

